\section{Format and Special \LaTeX{} Environments} 

You must write your seminar paper using a (scientific) word processing programme. \LaTeX{} is
strongly recommended. A style file will be provided to fulfil formatting requirements.
\cite{TUG} provides a great overview of the \LaTeX{} package, integrated development environments
(IDE), cheat sheets, and supplementary software for the \LaTeX{} novice. A very good (and free!)
overview of \LaTeX{} is provided in~\cite{Oet23}. 

If you wish to use a different word processing programme, please use Arial (font size 11 pt) or
Times New Roman (font size 12 pt) as font. Please ensure that you use the same font throughout the
document (header, footer, page numbers, and text). Use a line spacing of 1.3 for the general text.
The margins should be about 2.5 cm. The text should be formatted as justified text with hyphenation.
The length of the seminar paper is expected to be about 15 pages excluding table of contents,
references, and appendixes. According to the statutes of the Master's programme, the seminar paper
must be a minimum of 3000 and a maximum of 6000 words equivalent to 10--20 pages. Multilevel
numbered sections are to be used with a maximum depth of 3. Footnotes\footnote{This is a footnote}
should be avoided as they disrupt the reading flow and clash with the layout line at the bottom of
the page.

\subsection{How to Equations} %% <= This is a subsection (i.e., a section of order 2)
\LaTeX~is outstanding for mathematical typesetting and used extensively in the scientific research
community. You can write simple equations in line ${a^2 + b^2 = c^2}$. Alternatively, you may write
an equation as paragraph
\begin{equation}
    C = W \, \log_2 \left( 1 + \frac{P}{N_0 \, W} \right).
    \label{eq:cap}
\end{equation}
You may also write an equation in a new line without numbering
\[ a^2 + b^2 = c^2, \]
or alternatively mark the environment with a *
\begin{equation*}
    \sum_{k=1}^n k = \frac{n\,(n+1)}{(2}.
\end{equation*}
However, a number is great because you can reference to it once you provide a label, e.g., Equation
\eqref{eq:cap} depicts the famous channel capacity $C$ derived by Claude Shannon in \cite{Sha48}. It
is also possible to nicely set aligned systems of equations:
\begin{align}
    x + y & = 5 \\
    y & = 1. \\
\end{align}

Text within formulae (above all, formulae only consisting of text!) should be avoided and the
formula should rather be explained in the main text. If it is absolutely unavoidable, text should be
used as
\begin{verbatim}
     \text{text within an equation environment.}
\end{verbatim}
Spacing within formulas can be adapted with commands such as
\begin{verbatim}
    \quad or \,
\end{verbatim}
The following Equation~\eqref{quad} shows the impact of \verb|\quad|, \verb|\,|, and \verb|\text{abc}|:
\begin{equation}
    \label{quad} \forall x \in \{ z\in\mathbb{C} \,|\, z = a + j\,b, \ a \in \mathbb{R}, \text{ and }
    b = 0 \}: \quad x \in \mathbb{R}.
\end{equation}
Please check~\cite{TUG, Oet23} for much more on mathematical typesetting in \LaTeX{} as well as
helpful cheat sheets.

\subsection{How to Tables}
Table~\ref{tab:mytable} are also easily created. Make sure that the same font as in the main text is
also used within the tables. Each table must be referenced in the text (on the same or the following
page). Tables are typically placed on top of a page. The caption goes at the top.
\begin{table}[t]
    \caption{This is a table.}      %% caption of tables is on top of the table
    \label{tab:mytable}  
    \centering 
    \setlength{\tabcolsep}{4.5pt}
    \begin{tabular}{|l|c|r|}        %% align columns left, center, right; use vertical lines as separator
        \hline
        \textbf{Value 1} & \textbf{Value 2} & \textbf{Value 3}\\
        $\alpha$ & $\beta$ & $\gamma$ \\
        \hline               
        $1$ & $10$ & long \\
        $2$ & $20$ & short \\
        $3$ & $30$ & wide \\
        \hline 
    \end{tabular}
\end{table}

\subsection{How to Figures}
Images/Figures are also easily placed in \LaTeX. In Figure~\ref{fig:thilogo} the logo of THI is
depicted.  Place figures at the top of the page, unless it is very small and fits into the flow of
the paper. The caption goes below the figure. Each figure must be referenced in the main text. Do
not forget to properly cite the source if you have copied the figure. You need not state, however,
that the figure has been created by yourself.
\begin{figure}[t]
    \centering 
    \includegraphics[width=8em]{thi_logo_b_RGB.png}
    \caption{This is our THI logo.}     %% captions of figures below the figure
    \label{fig:thilogo}
\end{figure}

\subsubsection{TikZ} %% don't go deeper than subsubsections
Fonts in figures should be (approximately) the same size as used for the text in the body of the
paper. Figures need to be of camera--ready quality. As a general rule, use jpg format for photos and
png format for images with text and lines that you could not vectorise. You should always think
about not copying but redrawing a figure. 

A very powerful extension to \LaTeX{} is the TikZ environment which allows to create simple (and
very complex!) graphic elements. A brief example can be seen in Figure~\ref{fig:tikzsystem}.
\begin{figure}[t]
    \centering
    \begin{tikzpicture}
        [->, >=stealth', shorten >= 1pt, auto, node distance = 4cm, thick, inner sep = 3mm,
        box/.style = {rectangle, draw=black, very thick, align=center}, empty/.style = {rectangle}]

        %% place elements on canvas
        \node[empty] (input) at (0,0)              {~};
        \node[box] (system1) [right of = input]    {system 1}; 
        \node[box] (system2) [right of = system1]  {system 2};
        \node[empty] (output) [right of = system2] {~};

        %% connect elements
        \path (input)   edge node [inner sep = 1mm] {$x[n]$} (system1);
        \path (system1) edge node [inner sep = 1mm] {$y[n]$} (system2);
        \path (system2) edge node [inner sep = 1mm] {$z[n]$} (output);
    \end{tikzpicture}
    \caption{A simple system diagram using TikZ.}
    \label{fig:tikzsystem}
\end{figure}
Figure~\ref{fig:tikznn} depicts a neural network with three layers of nodes.
\begin{figure}[t]
    \centering
    \begin{tikzpicture}[->,
        line/.style = {draw = black,>=stealth'}]
        \foreach \i/\name in {0/Input, 1/Hidden, 2/Output}
            \foreach \j in {1,2,3}
                \path node[draw, circle, thick] (n\i-\j) at (\i*4,\j*4) {$\mathrm{\name}_\j$};
        \foreach \i in {0,1}
            \foreach \j in {1,2,3}
                \foreach \k in {1,2,3}
                    \path[line] (n\i-\j) -- (n\the\numexpr\i+1\relax-\k);
    \end{tikzpicture}
    \caption{Neural network.}
    \label{fig:tikznn}
\end{figure}

\subsection{How to Code}
It is also easy to add code in your paper as seen in Listing~\ref{lst:code} -- although it is
hardly ever required and beneficial. The perfect place for your code is either github or the Appendix.
\lstset{language=Python}
\lstset{frame=lines}
\lstset{caption={Insert code directly in your document.}}
\lstset{label={lst:code}}
\lstset{basicstyle=\footnotesize}
\begin{lstlisting}
    from brg.datastructures import Mesh
     
    mesh = Mesh.from_obj('faces.obj')
    mesh.draw()
\end{lstlisting}
